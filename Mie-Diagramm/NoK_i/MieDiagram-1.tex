\documentclass[tikz,crop]{standalone}

\usepackage{tikz,pgfplots}
\usepackage{physics}
%\usepackage{tikz-3dplot-circleofsphere}

\usepackage{tikz-3dplot} 
%
%\usetikzlibrary{decorations.shapes}
%\usetikzlibrary{arrows.meta,calc,decorations.markings,math,arrows.meta}
%\usetikzlibrary{%
%    decorations.pathreplacing,%
%    decorations.pathmorphing%
%}


%------------   Todo está en 2D porque no entiendo 3D   :c    ------------%


\begin{document}

%Angle Definitions
%-----------------

%set the plot display orientation
%synatax: \tdplotsetdisplay{\theta_d}{\varphi_d}
\tdplotsetmaincoords{60}{110}

%define polar coordinates for some vector
%TODO: look into using 3d spherical coordinate system
\pgfmathsetmacro{\rvec}{1.3}
\pgfmathsetmacro{\thetavec}{30}
\pgfmathsetmacro{\varphivec}{60}


\begin{tikzpicture}[scale=5, tdplot_main_coords]

%set up some coordinates 
	\coordinate (O) at (0,0,0);

%determine a coordinate (P) using (r,\theta,\varphi) coordinates.  This command
%also determines (Pxy), (Pxz), and (Pyz): the xy-, xz-, and yz-projections
%of the point (P).
%syntax: \tdplotsetcoord{Coordinate name without parentheses}{r}{\theta}{\varphi}
	\tdplotsetcoord{P}{\rvec}{\thetavec}{\varphivec}

%draw figure contents
%--------------------

%Draw the NP
	\draw[tdplot_screen_coords,ball color=yellow, opacity = 1] (O) circle (.25);
	\draw[color=blue, -, thick] (0,0,0) -- (.18,-.18,.18);	
	\node at (.09,-.09,.15){\color{blue} $a$};
	\node at (.25,-.25,.0){$n_m$};
	\node at (.01,-.18,-.15){$n_p$};		
	
%draw the main coordinate system axes
	\draw[thick,- latex] (0,0,0) -- (.8,0,0) node[anchor=north east]{$x$};
	\draw[thick,- latex] (0,0,0) -- (0,.8,0) node[anchor=north west]{$y$};
	\draw[thick,- latex] (0,0,0) -- (0,0,.8) node[anchor=south]{$z$};

%draw a vector from origin to point (P) 
	\draw[thick,color=green, - latex] (O) -- (P);
	\node at (1,.6,1.2) {\color{green} $\vb{r}$};
	
%draw projection on xy plane, and a connecting line
	\draw[dashed, color=green] (O) -- (Pxy);
	\draw[dashed, color=green] (P) -- (Pxy);
	\fill[green, opacity = .3] (O) --(Pxy)-- (P)--(O);	

%draw the angle \varphi, and label it
	%syntax: \tdplotdrawarc[coordinate frame, draw options]{center point}{r}{angle}{label options}{label}
	\tdplotdrawarc[- latex]{(O)}{0.6}{0}{\varphivec}{anchor=north}{$\varphi$}


%set the rotated coordinate system so the x'-y' plane lies within the
	%"theta plane" of the main coordinate system
	%syntax: \tdplotsetthetaplanecoords{\varphi}
	\tdplotsetthetaplanecoords{\varphivec}

%draw theta arc and label, using rotated coordinate system
	\tdplotdrawarc[tdplot_rotated_coords, - latex]{(0,0,0)}{0.6}{0}{\thetavec}{anchor=north east}{$\theta$}

% Plane Wave
	\foreach \i in {-3,...,3}{
		\draw[thick,tdplot_screen_coords,red, - latex] (\i/10,-1,0)--(\i/10,-.5,0);}
	\node[tdplot_screen_coords] at (4/10,-.5,0){\color{red}$\va{k}_i$};
	\node[tdplot_screen_coords] at (6/10,-.9,0){\color{red}Haz incidente};



	
\end{tikzpicture}

\end{document}